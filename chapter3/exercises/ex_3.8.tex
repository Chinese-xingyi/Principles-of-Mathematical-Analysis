% === Exercise 3.8 ===
\begin{Exercise}
\begin{proof}
Let $\sum a_n$ converges to $a$. Since $\{b_n\}$ is monotonic and bounded, by Theorem 3.14, $\{b_n\}$ converges to $b\in\mathbb{R}$. Since $\sum a_n$ converges, then its partial sums form a bounded sequence.

W.L.O.G., we assume $\{b_n\}$ is decreasing. Let $c_n := b_n-b$, then $\{c_n\}$ converges to $0$. By Theorem 3.42 (Dirichlet's Test), we obtain $\sum a_n c_n$ converges to $c\in\mathbb{R}$. Since
$$
\sum a_n c_n
= \sum a_n (b_n-b)
= \sum a_n b_n - b\sum a_n
= \sum a_n b_n - b a,
$$
then $\sum a_n b_n$ converges to $c+b a$. A similar argument establishes $\sum a_n b_n$ converges when $\{b_n\}$ is increasing.

Finally, we conclude $\sum a_n b_n$ converges.
\end{proof}
\end{Exercise}