% === Exercise 3.14 ===
\begin{Exercise}
\begin{enumerate}[(a)]
\item
\begin{proof}
Since $\lim_{n\to\infty} s_n = s$, then given $\epsilon>0$, there exists $N>0$ such that
$$
n\geq N \implies d(s_n,s) < \epsilon.
$$
This means
$$
n\geq N \implies s-\epsilon < s_n < s+\epsilon.
$$
Consider
$$
\sigma_n = \frac{\sum_{i=0}^{n} s_i}{n+1} = \frac{\sum_{i=0}^{N-1} s_i + \sum_{i=N}^{n} s_i}{n+1}.
$$
Since
\begin{alignat*}{7}
\quad&& \frac{\sum_{i=0}^{N-1}s_i + (n-N+1)(s-\epsilon)}{n+1} <& \sigma_n &<& \frac{\sum_{i=0}^{N-1}s_i + (n-N+1)(s+\epsilon)}{n+1} \\
\implies&& \frac{\sum_{i=0}^{N-1}s_i}{n+1}+\frac{n(s-\epsilon)}{n+1}-\frac{(N-1)(s-\epsilon)}{n+1} <& \sigma_n &<& \frac{\sum_{i=0}^{N-1}s_i}{n+1}+\frac{n(s-\epsilon)}{n+1}-\frac{(N-1)(s+\epsilon)}{n+1}.
\end{alignat*}
Taking the limit as $n\to\infty$, then
$$
s \leq \lim_{n\to\infty} \sigma_n \leq s.
$$
It follows from the Squeeze Theorem. Hence we conclude $\lim_{n\to\infty} \sigma_n = s$.
\end{proof}

\item
\begin{answer}
$s_n = (-1)^n$.
\end{answer}
\begin{proof}
Since 
$$
\frac{0}{n+1} \leq \sigma_n = \frac{\sum_{i=0}^{n} s_n}{n+1} \leq \frac{1}{n+1}.
$$
Taking the limit as $n\to\infty$, then $\lim_{n\to\infty} \sigma_n = 0$. However, $\{s_n\}$ diverges.
\end{proof}

\item
\begin{answer}
It can happen.
\end{answer}
\begin{proof}
Consider
$$
s_n = \begin{cases}
\left( \frac{1}{3} \right)^n + \sqrt[3]{n} & \mbox{if } n=k^3, k\in\mathbb{Z} \\
\left( \frac{1}{3} \right)^n & \mbox{otherwise}
\end{cases}.
$$
Notice that $\limsup_{n\to\infty} s_n = \infty$ and $s_n>0$ for all $n$. Then
\begin{align*}
\sum_{i=0}^{n}s_n
&= \sum_{i=0}^{n} \left( \frac{1}{3} \right)^i + \sum_{i=0}^{\lfloor \sqrt[3]{n} \rfloor}i \\
&\leq \sum_{i=0}^{\infty} \left( \frac{1}{3} \right)^i + \sum_{i=0}^{\lfloor \sqrt[3]{n} \rfloor}i \\
&= \frac{3}{2} + \frac{\sqrt[3]{n}(\sqrt[3]{n}+1)}{2} \\
&\leq \frac{3+\sqrt[3]{n}(\sqrt[3]{n}+\sqrt[3]{n})}{2} \\
&= \frac{3+2\sqrt[3]{n^2}}{2}.
\end{align*}
Since
$$
|\sigma_n|
= \left| \frac{\sum_{i=0}^{n}s_i}{n+1} \right|
\leq \frac{3+2\sqrt[3]{n^2}}{2(n+1)} \to 0\text{ as } n\to\infty,
$$
by the Squeeze Theorem, we obtain $\lim_{n\to\infty} \sigma_n = 0$ as promised.
\end{proof}
\end{enumerate}
\end{Exercise}