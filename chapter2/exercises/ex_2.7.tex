% === Exercise 2.7 ===
\begin{Exercise}
\begin{lemma}\label{lemma:ex_2.7a}
Let $E,F,G$ be subsets of a metric space. If $E=F\cup G$, then $\overline{E} \subseteq \overline{F}\cup\overline{G}$.
\end{lemma}
\begin{proof}
Let $x\in\overline{E}$, then $x\in E$ or $x\in E'$.

If $x\in E$, then $x\in F$ or $x\in G$ imply $x\in\overline{F}$ or $x\in\overline{G}$, respectively.
In either of two cases, we know $x\in \overline{F}\cup\overline{G}$.

Otherwise, if $x\in E'$, then $x\in F'$ or $x\in G'$. Suppose to contrary that $x\notin F'\cap G'$. Pick arbitrarily small radius $s$ and $t$ such that there is no point of $F$ in the neighborhood $N_s(x)$, and there is no point of $G$ in the neighborhood $N_t(x)$. Let $r := \min\{s,t\}$, since $N_r(x)\subseteq N_s(x)$ and $N_r(x)\subseteq N_t(x)$, then $N_r(x)$ contains no points of $F\cup G = E$. This contradicts the supposition that $x\in E'$. Hence $x\in F'$ or $x\in G'$. These imply $x\in \overline{F}$ or $x\in \overline{G}$, respectively. In either of two cases, we know $x\in \overline{F}\cup \overline{G}$.

Finally, no matter $x\in E$ or $x\in E'$, we know $x\in \overline{F}\cup \overline{G}$ always holds; hence we conclude $\overline{E} \subseteq \overline{F}\cup \overline{G}$.
\end{proof}

\begin{lemma}\label{lemma:ex_2.7b}
Let $E,F,G$ be subsets of a metric space. If $E=F\cup G$, then $\overline{E} \supseteq \overline{F}\cup\overline{G}$.
\end{lemma}
\begin{proof}
Let $x\in \overline{F}\cup \overline{G}$, then $x\in \overline{F}$ or $x\in \overline{G}$. In particular, $x\in F$ or $x\in F'$ or $x\in G$ or $x\in G'$.

If $x\in F$ or $x\in G$, then $x\in F\cup G$; hence $x\in \overline{F\cup G}$. This implies $x\in \overline{E}$.

Otherwise, if $x\in F'$ or $x\in G'$, then every neighborhood of $x$ contains an element of $F$ or $G$. Moreover, every element of $F$ or $G$ is an element of $F\cup G = E$. By definition, we see that $x\in E'$. Since $E' \subseteq E\cup E' = \overline{E}$, we have $x\in \overline{E}$.

Finally, in either of cases, we always know $x\in \overline{E}$; hence we conclude $\overline{E} \supseteq \overline{F}\cup\overline{G}$.
\end{proof}
\begin{enumerate}[a)]
\item
\begin{proof}
By the Lemma \ref{lemma:ex_2.7a} and the Lemma  \ref{lemma:ex_2.7b}, we know $\overline{E} = \overline{F}\cup\overline{G}$.

For $n=1$, if $B_1 = A_1$, then $\overline{B_1} = \overline{A_1}$ holds.

Suppose for $n=k$, 
$$
B_k = \bigcup_{i=1}^{k}A_i \implies \overline{B_k} = \bigcup_{i=1}^{k}\overline{A_i}.
$$
Then for $n=k+1$, if $B_{k+1} = \bigcup_{i=1}^{k+1}A_i$, then we have
\begin{align*}
\overline{B_{k+1}}
&= \overline{\cup_{i=1}^{k+1}A_i} \\
&= \overline{\left( \cup_{i=1}^{k}A_i \right) \cup A_{k+1}} \\
&= \overline{\cup_{i=1}^{k}A_i} \cup \overline{A_{k+1}} \\
&= \overline{B_k} \cup \overline{A_{k+1}} \\
&= \left( \cup_{i=1}^{k}\overline{A_i} \right) \cup \overline{A_{k+1}} \\
&= \cup_{i=1}^{k+1}\overline{A_i}
\end{align*}
also holds. 

By induction, we conclude the formula holds for all $n\in\mathbb{N}$.
\end{proof}

\item
\begin{proof}
From the Lemma \ref{lemma:ex_2.7a}, when we pick $r$ which is the minimum radius among all the candidates, $r$ exists if and only if the amount of candidates is finite. However, since $n=\infty$ which means candidates are infinite, then $r$ doesn't exist.

On the other hand, from the Lemma \ref{lemma:ex_2.7b}, it doesn't matter $n$ is finite or infinite; hence a similar argument from part a) establishes
$$
\overline{B} \supseteq \bigcup_{i=1}^{\infty}\overline{A_i}.
$$

\begin{itemize}
\item $\mathbf{Claim:this\ inclusion\ can\ be\ proper.}$
\end{itemize}

For example, from the Exercise 2.5, we let $A_i = \left\{ 1+\frac{1}{i+1} \right\}$ for all $i\in\mathbb{N}$ and $B=\cup_{i=1}^{\infty} A_i$. It follows that $\cup_{i=1}^{\infty}\overline{A_i} = \emptyset$ and $\overline{B} = \{1\}$; hence $\overline{B} \neq \bigcup_{i=1}^{\infty}\overline{A_i}$.

We conclude
$$
\overline{B} \supset \bigcup_{i=1}^{\infty}\overline{A_i}
$$
might occur, so this claim holds; furthermore, we complete the proof.
\end{proof}
\end{enumerate}
\end{Exercise}