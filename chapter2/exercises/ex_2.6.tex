% === Exercise 2.6 ===
\begin{Exercise}
	\begin{itemize}
		\item $\mathbf{Prove\ E'\ is\ closed.}$
		\begin{proof}
			By definition, it suffice to prove $(E')' \subseteq E'$. 
			
			Let $x\in (E')'$ arbitrarily. Now fix $x$ and pick arbitrarily small radius $r$ such that $s := \frac{r}{2}$ and $t := \frac{r}{4}$.
			
			Since $x\in (E')'$, there is a point $y\in E'$ in the neighborhood $N_s(x)$. And since $y\in E'$, there is also a point $z\in E$ in the neighborhood $N_t(y)$. 
			
			Notice that $y\neq z$ by the choice of radius $s$ and $t$.
			
			Consider
			$$
			d(x,z)
			\leq d(x,y) + d(y,z)
			< s + t
			= \frac{r}{2} + \frac{r}{4}
			< r.
			$$
			Hence for arbitrary radius $r$, there is a point $z\in E$ in the neighborhood $N_r(x)$, then we see that $x\in E'$. Moreover, $x$ is arbitrary, we know $(E')' \subseteq E'$ ; hence we conclude $E'$ is closed.
		\end{proof}
		
		\item $\mathbf{Prove\ E\ and\ E'\ have\ the\ same\ limit\ points.}$
		
		\begin{proof}
			We will claim $(\overline{E})' \subseteq E'$ and $E' \subseteq (\overline{E})'$ to obtain $(\overline{E})' = E'$ which is asked to prove.
			
			Let $x\in (\overline{E})'$ and arbitrary radius $r > 0$, then there is a point $y\in \overline{E}$ in the neighborhood $N_r(x)$.
			
			Since $\overline{E} = E\cup E'$, then if $y\in E$, this makes $x\in E'$ hold trivially; otherwise, if $y\in E'$, then $x\in E'$ holds from the previous proof. Hence we know $(\overline{E})' \subseteq E'$.
			
			On the other hand, if $x\in E'$, then every neighborhood of $x$ contains an element of $E$. This implies every neighborhood of $x$ contains an element of $\overline{E}$ trivially. Hence we know $E' \subseteq (\overline{E})'$.
			
			Finally, we conclude $(\overline{E})' = E'$ which means $E$ and $E'$ have the same limit points.
		\end{proof}
		
		\item $\mathbf{Do\ E\ and\ E'\ always\ have\ the\ same\ limit\ points?}$
		\begin{answer}
			No.
		\end{answer}
		\begin{proof}
			We give a counter-example.
			Let $E := \{0\} \cup \left\{ \frac{1}{n}:n\in\mathbb{N}\setminus\{1\} \right\}$, then $E' = \{0\}$ and $(E')' = \emptyset$. Hence $E' \neq (E')'$ which means $E$ and $E'$ don't have the same limit points.
		\end{proof}
	\end{itemize}
\end{Exercise}