% === Exercise 2.5 ===
\begin{Exercise}
	\begin{proof}
		Let $E := \left\{ 1+\frac{1}{n} : n\in \{2,3,\cdots\} \right\}$.
		\begin{itemize}
			\item \textbf{Claim only $1$ is a limit point.}
			
			By Archimedean Property, we can pick radius $r$ such that $\frac{1}{n} < r$. 
			Then $p_n = 1+\frac{1}{n}$ is a member of $E$ and $d(1, p_n) = \frac{1}{n} < r$. 
			Hence $1$ is a limit point.
			
			On the other hand, consider a point $x$ such that $x\notin E\cup \{1\}$. 
			We set 
			$$
			\delta = \min\{d(x,1), d(x,2), d(x, p_n), d(x, p_{n+1})\},
			$$
			and $r = \frac{\delta}{2}$. 
			Then there is no point of $E$ in $N_r(x)$.
		\end{itemize}
		
		Moreover, we set $F := \left\{ 2+\frac{1}{n} : n\in \{2,3,\cdots\} \right\}$. 
		And $G := \left\{ 3+\frac{1}{n} : n\in \{2,3,\cdots\} \right\}$. 
		A similar argument establishes only $2$ and $3$ are limit points.
		
		Now we set $K := E\cup F\cup G$, then $K$ is bounded by $4$.
		Hence we construct $K$ which is a bounded set of real numbers with exactly three limit points $1,2$ and $3$ as promised.
	\end{proof}
\end{Exercise}