% === Exercise 2.8 ===
\begin{Exercise}
	\begin{itemize}
		\item \textbf{Is every point of every open set $E \subset \mathbb{R}^2$ a limit point of $E$?}
		\begin{answer}
			Yes.
		\end{answer}
		\begin{proof}
			Let $x = (a, b)\in E$ arbitrarily. 
			Since $E$ is open, then $x\in E^{\circ}$; hence there is a neighborhood $N_r(x)$ where $r>0$ such that $N_r(x)\subset E$.
			
			Fix $x$, consider the neighborhood $N_s(x)$ for any radius $s>0$. 
			
			We can pick $y = \left( a+\frac{1}{2}\min\{r,s\}, b \right)$, then $d(x,y) = \frac{1}{2}\min\{r,s\} < r$. 
			This means there is a point $y\in N_r(x) \subset E$ in the neighborhood $N_s(x)$. 
			So every neighborhood of $x$ contains a point $y\neq x$ such that $y\in E$; hence $x\in E'$.
			
			Because $x$ was arbitrary, we conclude $E \subseteq E'$ for any open set $E\subset \mathbb{R}^2$.
		\end{proof}
		
		\item \textbf{Is every point of every closed set $E \subset \mathbb{R}^2$ a limit point of E?}
		\begin{answer}
			No.
		\end{answer}
		\begin{proof}
			For example, let $E := \left\{ (0,0) \right\}$, then $E' = \emptyset$; hence $(0,0)\notin E'$.
			
			Generally speaking, every finite subset $E$ of $\mathbb{R}^2$ is closed, however $E' = \emptyset$. 
		\end{proof}
	\end{itemize}
\end{Exercise}