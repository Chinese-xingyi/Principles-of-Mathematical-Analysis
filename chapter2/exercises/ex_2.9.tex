% === Exercise 2.9 ===
\begin{Exercise}
	\begin{enumerate}[a)]
		\item
		\begin{proof}
			Let $x\in E^{\circ}$ arbitrarily, then there is a neighborhood $N_r(x)\subset E$ with $r>0$.
			
			Let $y\in N_r(x)$ arbitrarily, then we have $d(x,y)<r$. 
			Pick a radius $s=r-d(x,y)>0$. It follows that $N_s(y)\subset E$. 
			
			Let $z\in N_s(y)$ arbitrarily, since
			$$
			d(x,z) \leq d(x,y)+d(y,z) < d(x,y) + s = r,
			$$
			there is a neighborhood $N_s(y)\subset E$ with $s>0$. 
			Because $z$ was arbitrary, then we have $y\in E^{\circ}$.
			
			Moreover, $y$ was arbitrary, then $N_r(x)\subset E^{\circ}$ which means $x\in (E^{\circ})^{\circ}$; also, by the choice of $x$ arbitrarily, hence $E^{\circ}$ is open.
		\end{proof}
		
		\item
		\begin{proof}
			$(\Longrightarrow)$
			Since $E$ is open, then every point of $E$ is an interior point of $E$; hence $E^{\circ} = E$.
			
			$(\Longleftarrow)$
			Since $E = E^{\circ}$, and we know $E^{\circ}$ is always open by the Exercise 2.9a, then $E$ is open.
		\end{proof}
		
		\item
		\begin{proof}
			Let $x\in G$ arbitrarily. 
			Since $G$ is open, we know $x\in G^{\circ}$. 
			Then there is a neighborhood $N_r(x)\subset G$ with $r>0$. 
			Because $G\subset E$ by hypothesis, $N_r(x)\subset E$ which means $x\in E^{\circ}$. 
			By the arbitrary choice of $x$, we conclude $G\subset E^{\circ}$.
		\end{proof}
		
		\item
		\begin{proof}
			We need to prove $(E^{\circ})^c = \overline{E^c}$ which means $(E^{\circ})^c \subseteq \overline{E^c}$ and $(E^{\circ})^c \supseteq \overline{E^c}$.
			
			\begin{itemize}
				\item $\mathbf{Claim:\ (E^{\circ})^c \subseteq \overline{E^c}}$
				
				Let $x\in(E^{\circ})^c$ arbitrarily, then $x\notin E^{\circ}$.
				This implies every neighborhood of $x$ contains no elements in $E$; hence every neighborhood of $x$ contains some elements $y\in E^c$.
				
				If $x=y$ for at least one of these neighborhoods, then we know $x\in E^c$ because $x=y$ and $y\in E^c$.
				
				Otherwise, if $x\neq y$ for all neighborhoods, then by definition of a limit point, we know $x\in (E^c)'$.
				
				In either of two cases, $x$ is always a member of $E^c \cup (E^c)' = \overline{E^c}$. 
				
				By the arbitrary choice of $x$, we conclude $(E^{\circ})^c \subseteq \overline{E^c}$.
				
				\item $\mathbf{Claim:\ (E^{\circ})^c \supseteq \overline{E^c}}$
				
				Let $x\in\overline{E^c}$ arbitrarily, then $x\in E^c$ or $x\in (E^c)'$.
				
				If $x\in E^c$, then $x\notin E$. 
				This implies that every neighborhood of $x$ is not contained in $E$, so $x\notin E^{\circ}$; hence $x\in (E^{\circ})^c$.
				
				Otherwise, if $x\in (E^c)'$, then every neighborhood $N_r(x)$ contains a point $y\in E^c$ with $r>0$. 
				Since
				$$
				d(x,x) = 0 < d(x,y) < r,
				$$
				we know $x\in(E^{\circ})^c$.
				
				In either of two cases, $x\in(E^{\circ})^c$ always; also, by arbitrary choice of $x$, we conclude $(E^{\circ})^c \supseteq \overline{E^c}$.
			\end{itemize}
			Finally, we together two claims to obtain $(E^{\circ})^c = \overline{E^c}$.
		\end{proof}
		
		\item
		\begin{answer}
			No.
		\end{answer}
		\begin{proof}
			Let $E := (0,1) \cup (1,2) \subset \mathbb{R}$, then $\overline{E} = [0,2]$; hence $E^{\circ} = E$ and $\overline{E}^{\circ} = (0,2)$.
			This implies $E^{\circ} \neq \overline{E}^{\circ}$ as a counter-example.
		\end{proof}
		
		\item
		\begin{answer}
			No.
		\end{answer}
		\begin{proof}
			Let $E := \mathbb{Q}$ which is the set of all rational numbers, then $E^{\circ} = \emptyset$; hence $\overline{E} = \mathbb{R}$ and $\overline{E^{\circ}} = \emptyset$. 
			This implies $\overline{E}\neq \overline{E^{\circ}}$ as a counter-example.
		\end{proof}
	\end{enumerate}
\end{Exercise}