% === Exercise 5.11 ===
\begin{Exercise}
	\begin{proof}
		Consider
		$$
		\lim_{h\to 0}\frac{f(x+h) + f(x-h) -2f(x)}{h^2}.
		$$
		We observe
		\begin{align*}
		&\lim_{h\to 0}f(x+h)+f(x-h)-2f(x) = 0; \\
		&\lim_{h\to 0}h^2 =0.
		\end{align*}
		By L'Hospital's Rule, we differentiate them over $h$, then obtain
		\begin{align*}
		\lim_{h\to 0}\frac{f(x+h) + f(x-h) -2f(x)}{h^2}
		&= \lim_{h\to 0}\frac{f'(x+h) - f'(x-h)}{2h} \\
		&= \frac{1}{2} \lim_{h\to 0}\left( \frac{f'(x+h)-f'(x)}{h} + \frac{f'(x) - f'(x-h)}{h} \right)\\
		&= \frac{1}{2} \big(f''(x) + f''(x) \big) \\
		&= f''(x)
		\end{align*}
		which is a desired result.
	\end{proof}
	\begin{itemize}
		\item \textbf{An example that the limit may exist even if $f''(x)$ does not.}
	\end{itemize}
	\begin{answer}
		$f(x) = \begin{cases}
		x^2 & \mbox{for } x > 0 \\
		-x^2 & \mbox{for } x < 0 \\
		0 & \mbox{otherwise}
		\end{cases}$.
	\end{answer}
	\begin{proof}
		Since $f''(x) = \begin{cases}
		2 & \mbox{for } x> 0 \\
		-2 & \mbox{for } x<0 \\
		\end{cases}$, then $f''(x)$ does not exist at $x=0$.
	\end{proof}
\end{Exercise}