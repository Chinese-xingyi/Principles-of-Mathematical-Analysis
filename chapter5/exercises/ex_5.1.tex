% === Exercise 5.1 ===
\begin{Exercise}
\begin{proof}
Consider
$$
|f(x)-f(y)| \leq (x-y)^2 = |x-y|^2.
$$
For $x\neq y$, we have
$$
\left| \frac{f(x)-f(y)}{x-y} \right| \leq |x-y|.
$$
Taking the limit on both sides as $x\to y$, we obtain
$$
|f'(y)| \leq 0.
$$
This means $f'(y) = 0$. Since $y$ was arbitrary, then $f'(y) = 0$ for all $y\in\mathbb{R}$. Hence by Theorem 5.11(b), $f$ is constant.
\end{proof}
\end{Exercise}