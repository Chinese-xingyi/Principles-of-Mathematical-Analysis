% === Exercise 5.25 ===
\begin{Exercise}
	\begin{enumerate}[(a)]
		\item 
		\begin{solution}
			Consider the tangent line to a point $(x_n, y_n)$ is
			$$
			y-y_n = f'(x_n) (x-x_n).
			$$
			Pick $x = x_{n+1}$ and $y = 0$, also put $y_n = f(x_n)$, then we have
			\begin{equation}\label{eq:ex_5.25a}
			x_{n+1} = x_n - \frac{f(x_n)}{f'(x_n)}.
			\end{equation}
			Hence in the geometrical opinion, $x_{n+1}$ is chosen to be the intersection of the tangent line to $\big(x_n, f(x_n)\big)$ and $x$-axis.
		\end{solution}
		
		\item
		\begin{proof}
			By hypothesis and Intermediate Value Theorem, we know $f(x) > 0$ when $x > \xi$. This implies
			$$
			\frac{f(x_n)}{f'(x_n)} > 0.
			$$
			We observe $x_{n+1} < x_n$ by \eqref{eq:ex_5.25a}.
			
			Since $f''(x) \geq 0$, then $f'(x)$ is increasing. This means
			$$
			c < x_n \implies f'(c) \leq f'(x_n).
			$$
			Pick $c$ such that $\xi < c < x_n$, then by the Mean Value Theorem, we have
			$$
			\frac{f(x_n)-f(\xi)}{x_n-\xi} = f'(c) \leq f'(x_n).
			$$
			Notice that $f(\xi) = 0$, then rearrange the inequality to obtain
			$$
			x_n - \frac{f(x_n)}{f'(x_n)} = x_{n+1}\geq \xi.
			$$
			We have $\xi < x_{n+1} < x_n$ so far.
			
			Construct the sequence $\{x_n\}$.
			Since $x_n$ is decreasing and bounded below by $\xi$, by Theorem 3.14, $\{x_n\}$ converges to some $\eta$.
			Consider \eqref{eq:ex_5.25a}.
			Taking the limit on both sides as $n\to\infty$, we have
			$$
			\eta = \eta - \lim_{n\to\infty}\frac{f(x_n)}{f'(x_n)}.
			$$
			This implies
			$$
			\lim_{n\to\infty}\frac{f(x_n)}{f'(x_n)} = 0.
			$$
			Since $f''(x)$ is bounded and so is $f'(x)$, we obtain
			$$
			\lim_{n\to\infty}f(x_n) = 0 = f(\xi).
			$$
			Notice that $f'(x)$ exists on $[a,b]$, and hence $f$ is continuous on $[a,b]$. 
			By Theorem 4.2, we conclude that
			$$
			\lim_{n\to\infty}x_n = \xi
			$$
			as promised eventually.
		\end{proof}
		
		\item
		\begin{proof}
			Using the Taylor's Theorem, we have
			$$
			f(\xi) = f(x_n) + f'(x_n)(\xi-x_n)+\frac{f''(t_n)}{2}(\xi-x_n)^2
			$$
			where $t_n\in(\xi, x_n)$.
			Since $f'(x_n)> 0$, divides $f'(x_n)$, then we have
			$$
			\frac{f(\xi)}{f'(x_n)} = \frac{f(x_n)}{f'(x_n)} + (\xi-x_n) + \frac{f''(t_n)}{2f'(x_n)}(\xi-x_n)^2.
			$$
			Since $f(\xi) = 0$, we rearrange the equation and obtain
			$$
			x_n - \frac{f(x_n)}{f'(x_n)} - \xi = \frac{f''(t_n)}{2f'(x_n)}(x_n-\xi)^2.
			$$
			By definition of $x_{n+1}$, we conclude
			\begin{equation}\label{eq:ex_5.25d}
			x_{n+1} - \xi = \frac{f''(t_n)}{2f'(x_n)}(x_n-\xi)^2.
			\end{equation}
		\end{proof}
		
		\item
		\begin{proof}
			We have known $0 \leq x_{n+1} - \xi$ from part(b) trivially.
			Consider \eqref{eq:ex_5.25d}.
			Since $f''(x) \leq M$ and $f'(x) \leq \delta$ by hypothesis, we have
			$$
			x_{n+1}-\xi \leq \frac{M}{2\delta}(x_n-\xi)^2 = A(x_n-\xi)^2.
			$$
			This inequality holds for $n\in\mathbb{N}$. By induction, we know
			\begin{align*}
			x_{n+1}-\xi
			&\leq A(x_n-\xi)^2 \\
			&\leq A\cdot A^2 (x_{n-1}-\xi)^4 \\
			&\leq \cdots \\
			&\leq A\cdot A^2 \cdots \cdot A^{2^{n-1}} (x_1-\xi)^{2^n} \\
			&= A^{2^n-1} (x_1-\xi)^{2^n} \\
			&= \frac{1}{A} \big[A(x_1-\xi)\big]^{2^n}.
			\end{align*}
			Hence we deduce
			$$
			0 \leq x_{n+1}-\xi \leq \frac{1}{A} \big[A(x_1-\xi)\big]^{2^n}.
			$$
		\end{proof}
		
		\item
		\begin{itemize}
			\item \textbf{Show that Newton's method amounts to finding a fixed point of the function $g$.}
		\end{itemize}
		\begin{solution}
			Suppose we have known $\eta$ is a fixed point of the function $g$ such that $g(\eta) = \eta$. Then
			$$
			g(\eta) - \eta = \frac{f(\eta)}{f'(\eta)} = 0.
			$$
			Since $f''$ is bounded and so is $f'$, then we have $f(\eta) = 0$.
			This means $\eta$ is the unique point such that $f(\eta) = 0$.
		\end{solution}
		\begin{itemize}
			\item How does $g'(x)$ behave for $x$ near $\xi$?
		\end{itemize}
		\begin{solution}
			Notice that $g'$ exists on $[a,b]$. Compute
			$$
			g'(x) = 1 - \frac{\big[ f'(x) \big]^2 - f''(x) f(x)}{\big[ f'(x) \big]^2} = \frac{f''(x) f(x)}{\big[ f'(x) \big]^2}.
			$$
			Taking the limit on both sides as $x\to xi$, we have
			$$
			\lim_{x\to\xi}g'(x) = \lim_{x\to\xi}f(x) \lim_{x\to\xi} \frac{f''(x)}{[f'(x)]^2}.
			$$
			Since $f$ is continuous near $\xi$, then we know
			$$
			\lim_{x\to\xi} f(x) = f(\xi) = 0.
			$$
			This means
			$$
			\lim_{x\to\xi}g'(x) = 0.
			$$
			We deduce $g'(x)$ tends to $0$ as $x$ tends $\xi$.
		\end{solution}
		
		\item
		\begin{solution}
			Compute
			$$
			f'(x) = \frac{1}{3} x^{-\frac{2}{3}}.
			$$
			We observe $0$ is a fixed point such that $f(0) = 0$.
			However, $f'(0)$ does not exist.
			Hence the sequence $\{x_n\}$ diverges.
		\end{solution}
	\end{enumerate}
\end{Exercise}