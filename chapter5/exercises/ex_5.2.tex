% === Exercise 5.2 ===
\begin{Exercise}
	\begin{itemize}
		\item \textbf{Prove $f$ is strictly increasing in $(a,b)$.}
	\end{itemize}
	\begin{proof}
		Since $f'(x) > 0$ in $(a,b)$, then $f$ is differentiable on $(a,b)$. 
		By the Mean Value Theorem, we consider
		$$
		f(x_1) - f(x_2) = (x_1 - x_2) f'(x)
		$$
		for $x_1,x_2\in (a,b)$. This implies
		$$
		f'(x) = \frac{f(x_1)-f(x_2)}{x_1-x_2} > 0.
		$$
		Then $x_1 > x_2$ if and only if $f(x_1) > f(x_2)$. This means $f$ is strictly increasing in $(a,b)$.
	\end{proof}
	\begin{itemize}
		\item \textbf{Prove that $g$ is differentiable and the formula holds.}
	\end{itemize}
	\begin{proof}
		We have already known $f$ is strictly increasing in $(a,b)$, then $f$ is one-to-one and continuous on $(a,b)$. 
		Hence $g = f^{-1}$ exists and $g\big( f(x) \big) = x$ for all $x\in(a,b)$; also $f(x)\to f(c)$ as $x\to c$ where $c\in(a,b)$.
		
		Now we should prove $g$ is differentiable on $f(a,b)$ so that we can differentiate it over $x$ to get the formula.
		Consider
		$$
		f'(x) = \lim_{x\to c}\frac{f(x)-f(c)}{x-c}
		= \lim_{f(x)\to f(c)}\frac{f(x)-f(c)}{g\big( f(x) \big) - g\big( f(c) \big)}
		= \frac{1}{g'(f(x))}
		$$
		where $c\in(a,b)$. This conclude
		$$
		g'\big( f(x) \big) = \frac{1}{f'(x)}
		$$
		for $x\in(a,b)$.
	\end{proof}
\end{Exercise}