% === Exercise 5.22 ===
\begin{Exercise}
	\begin{enumerate}[(a)]
		\item 
		\begin{proof}
			Suppose to contrary that $f$ has more than one fixed point, says $x_1$, $x_2$.
			Notice that $x_1\neq x_2$.
			Then by Mean Value Theorem, we consider
			$$
			f'(t) = \frac{f(x_1)-f(x_2)}{x_1-x_2}
			= \frac{x_1-x_2}{x_1-x_2}
			=1
			$$
			where $t\in (x_1,x_2)$.
			This contradicts $f'(t) \neq 1$ for all $t$.
			Hence $f$ has at most one fixed point.
		\end{proof}
		
		\item
		\begin{proof}
			We observe
			$$
			f'(t) = 1-\frac{e^t}{(1+e^t)^2}.
			$$
			This means $0<f'(t)<1$ for all $t$.
			Suppose to contrary there exists a fixed point $t_0\in\mathbb{R}$ such that $f(t_0) = t_0$.
			Then $f(t_0) = t_0 + (1+e^t_0)^{-1}$ implies $e^{t_0} = 0$.
			There is no $t_0\in\mathbb{R}$ satisfied so is a contradiction.
			It follows that $f$ has no fixed point.
		\end{proof}
		
		\item
		\begin{proof}
			Since $|f'(t)| \leq A$ for all $t$, then
			$$
			\left| \frac{f(x_{n+1})-f(x_n)}{x_{n+1}-x_n} \right | \leq A
			$$
			which implies
			$$
			\left| f(x_{n+1})-f(x_n) \right| \leq A \left| x_{n+1}-x_n \right|.
			$$
			This inequality holds for all $n\in\mathbb{N}$. By induction, we have
			$$
			|f(x_{n+1})-f(x_n)| \leq A^n | x_2-x_1 |.
			$$
			Notice that $A < 1$, for $k > 0$, we consider
			\begin{align*}
			| x_{n+k} - x_n |
			&\leq |x_{n+k}-x_{n+k-1}| + |x_{n+k-1}-x_{n+k-2}| + \cdots + | x_2 - x_1 | \\
			&= A^{n+k-2} |x_2-x_1| + A^{n+k-3} |x_2-x_1| + \cdots + A^{n-1} |x_2 - x_1| \\
			&\leq k A^{n-1} | x_2-x_1 |.
			\end{align*}
			Taking the limit on both sides as $n\to\infty$, we have
			$$
			\lim_{n\to\infty} |x_{n+k}-x_n| \leq 0.
			$$
			Hence $\lim_{n\to\infty}|x_{n+k}-x_n| = 0$, it follows that the sequence $\{x_k\}$ is Cauchy.
			Since $\mathbb{R}$ is compact, then by Theorem 3.11(b), we know there is $x\in\mathbb{R}$ such that 
			$$
			\lim_{n\to\infty} x_n = x.
			$$
			Moreover, by definition of $f(x_n)$, we also have 
			$$
			\lim_{n\to\infty} f(x_n) = x.
			$$
			Since $f'(t)$ exists for all $t\in\mathbb{R}$ by hypothesis, then $f$ is continuous on $\mathbb{R}$, we conclude
			$$
			f(x) = x
			$$
			which means $x$ is a fixed point of $f$.
		\end{proof}
		
		\item
		\begin{solution}
			Consider the points $\big(x_n, f(x_n)\big)$.
			We know when $n\to\infty$, the point must be a fixed point $(x, x)$.
			If we start with $\big(x_1, f(x_1)\big)$, then we want to arrive at $\big(x_2, f(x_2)\big)$. We just pick the path
			$$
			\big(x_1, f(x_1)\big) \to \big(x_2, f(x_1)\big) \to \big(x_2, f(x_2)\big)
			$$
			which means
			$$
			\big(x_1, x_2\big) \to \big(x_2, x_2\big) \to \big(x_2, x_3\big).
			$$
			Then we arrive at $\big(x_3, f(x_3)\big)$, which means
			
			$$
			\big(x_2, x_3\big) \to \big(x_3, x_3\big) \to \big(x_3, x_4\big).
			$$
			This process continue until we arrive at the fixed point.
			By induction, we obtain the zig-zag path.
		\end{solution}
	\end{enumerate}
\end{Exercise}