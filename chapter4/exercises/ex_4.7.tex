% === Exercise 4.7 ===
\begin{Exercise}
	\begin{itemize}
		\item Prove that $f$ is bounded on $\mathbb{R}^2$.
	\end{itemize}
	\begin{proof}
		For $(x,y) \neq (0,0)$, we consider $(x-y^2)^2 \geq 0$. Then
		\begin{alignat*}{7}
		\quad&& 0 &\leq \left| (x-y^2)^2 \right| \\
		\implies&& 0 &\leq \left| x^2 - 2x y^2+ y^4 \right| \leq |x^2+y^4|+|2x y^2| \\
		\implies&& \left| \frac{2 x y^2}{x^2+y^4} \right| &\leq 1 \\
		\implies&& |f(x,y)| &\leq \frac{1}{2}
		\end{alignat*}
		Otherwise, for $(x,y)=(0,0)$, we have $|f(0,0)| = 0 \leq \frac{1}{2}$ trivially.
		
		Hence $f(x,y)$ is bounded by $\frac{1}{2}$.
	\end{proof}
	
	\begin{itemize}
		\item Prove that $g$ is unbounded in every neighborhood of $(0,0)$.
	\end{itemize}
	\begin{proof}
		Let $x := t^3$ and $y := t$, we observe $\lim_{t\to 0} x = 0$ and $\lim_{t\to 0} y = 0$. Then for $t\neq 0$, we know $(x,y) \neq (0,0)$. Consider
		$$
		\lim_{t\to 0} g(x,y) = \lim_{t\to 0} g(t^3,t) = \lim_{t\to 0}\frac{1}{t} = \infty.
		$$
		Hence $g$ is unbounded in every neighborhood of $(0,0)$.
	\end{proof}
	
	\begin{itemize}
		\item Prove that $f$ is not continuous at $(0,0)$.
	\end{itemize}
	\begin{proof}
		Given $\epsilon = \frac{1}{2}$, for $(x,y)\neq (0,0)$, there exists $\delta > 0$ such that
		$$
		d( (0,0),(x,y) ) <\delta \implies d(f(0,0),f(x,y))\geq \epsilon.
		$$
		Regardless of $\delta$, we just pick $x=y^2$, then
		$$
		d(f(0,0),f(x,y)) = \frac{y^4}{2y^4} = \frac{1}{2} \geq \epsilon.
		$$
		Hence $f$ is not continuous at $(0,0)$.
	\end{proof}
	
	\begin{itemize}
		\item The restrictions of both $f$ and $g$ to every straight line in $\mathbb{R}^2$ are continuous. 
	\end{itemize}
	\begin{proof}
		Since a straight line which doesn't pass through $(0,0)$ is always continuous trivially, then here we only prove the lines passing through $(0,0)$ are continuous.
		
		Consider the line $y = c x$ where $c$ is constant. Given $\epsilon>0$, we observe
		$$
		d((0,0), (x,y))
		= \sqrt{x^2+y^2} 
		= \sqrt{x^2 + c^2 x^2}
		= |x| \sqrt{c^2+1}
		< \delta.
		$$
		Put
		$$
		\delta = \frac{\sqrt{c^2+1}}{c^2}\epsilon.
		$$
		For $x\neq 0$, this implies
		$$
		d(f(0,0), f(x,y))
		= \left| \frac{x y^2}{x^2+y^4} \right|
		= \left| \frac{c^2 x^3}{x^2+c^4 x^4} \right|
		= \left| \frac{c^2 x}{1+c^4 x^2} \right|
		< |c^2 x|
		< c^2 \frac{\delta}{\sqrt{c^2+1}}
		= \epsilon.
		$$
		So $f$ is continuous on this line.
		
		Then we consider the line $x=0$. Given $\epsilon>0$, for $y\neq 0$, we observe
		$$
		d(f(0,0),f(x,y))
		= \left| \frac{x y^2}{x^2+y^4} \right|
		= \frac{0}{y^4}
		= 0
		< \epsilon
		$$
		regardless of the choice of $\delta$ such that $d((0,0),(x,y))<\delta$. Hence $f$ is continuous on this line.
		
		Finally, we conclude the restriction of $f$ to every straight line in $\mathbb{R}^2$ is continuous. A similar argument proves the restriction of $g$ to every straight line in $\mathbb{R}^2$ is also continuous.
	\end{proof}
\end{Exercise}