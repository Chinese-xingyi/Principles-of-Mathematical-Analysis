% === Exercise 4.2 ===
\begin{Exercise}
\begin{proof}
If $f(\overline{E})$ is empty, then it holds trivially. Suppose $f(\overline{E})$ is nonempty. Let $y\in f(\overline{E})$. Put $y=f(e)$ for some $e\in E\cup E'$.

If $e\in E$, then $y\in f(E)$. This implies $y\in\overline{f(E)}$ immediately.

Otherwise, if $e\in E'$, then $N_{\delta}(e)$ contains infinitely many points in $E$ with $\delta>0$. Since $f$ is continuous, then given $\epsilon>0$, we know for all $\delta>0$,
$$
x\in N_{\delta}(e) \implies f(x)\in N_{\epsilon}(y).
$$
It follows that $N_{\epsilon}(y)$ contains infinitely many points in $f(E)$. Hence $y$ is a limit point of $f(E)$.

In either of two cases, we observe $y\in f(E)\cup f(E)' = \overline{f(E)}$. By the arbitrary choice of $y$, we conclude $f(\overline{E})\subset \overline{f(E)}$.
\end{proof}

\begin{itemize}
\item $f(\overline{E})$ can be a proper subset of $\overline{f(E)}$.
\end{itemize}
\begin{solution}
We define $f:\mathbb{Z}\to\mathbb{R}$ by $f(x) = \frac{1}{x}$. Pick $\delta$ such that $0 < \delta < 1$. Given $\epsilon>0$, for all $x,y\in \mathbb{Z}$, we have
$$
d(x,y)< \delta \implies x = y.
$$
Hence $d(f(x),f(y)) = 0 < \epsilon$. This means $f$ is uniformly continuous and so is continuous.

Now we put $E = \mathbb{Z}$. We see that
$$
f(\overline{\mathbb{Z}}) 
= f(\mathbb{Z}) 
= \{ \frac{1}{x} : x\in\mathbb{Z} \}
\subset \{ \frac{1}{x} : x\in\mathbb{Z} \} \cup \{0\}
= \overline{f(\mathbb{Z})}.
$$
\end{solution}
\end{Exercise}