% === Exercise 6.5 ===
\begin{Exercise}
	\begin{itemize}
		\item \textbf{Does it follow that $f\in\mathfrak{R}$ if $f^2\in\mathfrak{R}$?}
	\end{itemize}
	\begin{answer}
		No.
	\end{answer}
	\begin{proof}
		Consider
		$$
		f(x) = \begin{cases}
		1 & \mbox{for } x\in\mathbb{Q} \\
		-1 & \mbox{otherwise} 
		\end{cases}.
		$$
		A similar argument from Exercise 6.4 proves $f\notin\mathfrak{R}$.
		However, 
		$$
		f^2(x) = 1
		$$
		for all $x\in\mathbb{R}$ shows $f^2\in\mathfrak{R}$ on $[a,b]$ trivially.
	\end{proof}
	
	\begin{itemize}
		\item \textbf{Does it follow that $f\in\mathfrak{R}$ if $f^3\in\mathfrak{R}$?}
	\end{itemize}
	\begin{answer}
		Yes.
	\end{answer}
	\begin{proof}
		Since $f$ is bounded by hypothesis, we can suppose $m \leq f \leq M$ for some $m,M\in\mathbb{R}$.
		Define a function $\phi$ by
		$$
		\phi(x) = \sqrt[3]{f(x)}.
		$$
		We can easily see that $\phi$ is continuous on $\mathbb{R}$ so is on $[m,M]$.
		Since
		$$
		f(x) = \phi\big( f^3(x) \big)
		$$
		for all $x\in[a,b]$, by Theorem 6.11, it follows that $f\in\mathfrak{R}$ on $[a,b]$.
	\end{proof}
\end{Exercise}