% === Exercise 9.27 ===
\begin{Exercise}
	\begin{enumerate}[(a)]
		\item
		We prove three functions are continuous in $\mathbb{R}^2$ one by one.
		\begin{itemize}
			\item $f$ is continuous in $\mathbb{R}^2$.
		\end{itemize}
		\begin{proof}
			$f$ is continuous on $\mathbb{R}^2\setminus (0,0)$ trivially, so we consider for $(x,y) = (0,0)$. 
			
			Given $\epsilon>0$, pick $\delta = \sqrt{2\epsilon}$ such that
			$$
			\| (x,y) - (0,0) \| = \sqrt{x^2+y^2} < \delta
			$$
			implies
			\begin{align*}
			|f(x,y)-f(0,0)|
			&= \left| \frac{xy(x^2-y^2)}{x^2+y^2} \right| \\
			&\leq \frac{1}{2}|x^2-y^2| & (\because 2|xy| \leq x^2+y^2) \\
			&\leq \frac{1}{2} |x^2+y^2| \\
			&< \epsilon.
			\end{align*}
			This means $f$ is continuous at $(0,0)$.
			Hence we conclude $f$ is continuous in $\mathbb{R}^2$.
		\end{proof}
		
		\begin{itemize}
			\item $D_1 f$ is continuous in $\mathbb{R}^2$. 
		\end{itemize}
		\begin{proof}
			Notice that 
			$$
			D_1 f(0,0)
			= \lim_{h\to 0} \frac{f(h,0)-f(0,0)}{h}
			= 0.
			$$
			Calculate
			\begin{align*}
			D_1 f(x,y) = \frac{x^4 y + 4x^2 y^3 - y^5}{(x^2+y^2)^2}, && (x,y)\neq(0,0).
			\end{align*}
			$D_1 f$ is continuous on $\mathbb{R}^2\setminus (0,0)$ trivially, so we consider for $(x,y) = (0,0)$.
			
			Observe that
			\begin{align*}
			|D_1 f(x,y)-D_1 f(0,0)|
			&= \left| \frac{x^4 y + 4x^2 y^3 - y^5}{(x^2+y^2)^2} \right| \\
			&= \left| \frac{y(x^2-y^2)}{x^2+y^2} + \frac{4x^2y^3}{(x^2+y^2)^2} \right| \\
			&\leq \left| \frac{y(x^2-y^2)}{x^2+y^2} \right| + \left| \frac{4x^2y^3}{(x^2+y^2)^2} \right| \\
			&\leq |y| + \left| \frac{4x^2y^3}{(x^2+y^2)^2} \right| & (\because |x^2-y^2| \leq |x^2+y^2|) \\
			&\leq |y| + |y| & (\because 2|xy|\leq x^2+y^2) \\
			&= 2|y| \to 0 \text{ as } (x,y) \to (0,0).
			\end{align*}
			Since $D_1 f(0,0) = 0 = \lim_{(x,y)\to(0,0)}$, then $D_1 f$ is continuous at $(0,0)$.
			Hence we conclude $D_1 f$ is continuous in $\mathbb{R}^2$.
		\end{proof}
		
		\begin{itemize}
			\item $D_2 f$ is continuous in $\mathbb{R}^2$.  
		\end{itemize}
		\begin{proof}
			Notice that 
			$$
			D_2 f(0,0) 
			= \lim_{k\to 0} \frac{f(0,k) - f(0,0)}{k}
			= 0.
			$$
			Calculate
			\begin{align*}
			D_2 f(x,y) = \frac{x^5-4x^3y^2-xy^4}{(x^2+y^2)^2}, && (x,y)\neq(0,0).
			\end{align*}
			$D_2 f$ is continuous on $\mathbb{R}^2\setminus (0,0)$ trivially, so we consider for $(x,y) = (0,0)$.
			
			Observe that
			\begin{align*}
			|D_2 f(x,y)-D_2 f(0,0)|
			&= \left| \frac{x^5-4x^3y^2-xy^4}{(x^2+y^2)^2} \right| \\
			&= \left| \frac{x(x^2-y^2)}{x^2+y^2} - \frac{4x^3y^2}{(x^2+y^2)^2} \right| \\
			&\leq \left| \frac{x(x^2-y^2)}{x^2+y^2} \right| + \left| \frac{4x^3y^2}{(x^2+y^2)^2} \right| \\
			&\leq |x| + \left| \frac{4x^3y^2}{(x^2+y^2)^2} \right| & (\because |x^2-y^2| \leq |x^2+y^2|) \\
			&\leq |x| + |x| & (\because 2|xy|\leq x^2+y^2) \\
			&= 2|x| \to 0 \text{ as } (x,y) \to (0,0).
			\end{align*}
			Since $D_2 f(0,0) = 0 = \lim_{(x,y)\to(0,0)}$, then $D_2 f$ is continuous at $(0,0)$.
			Hence we conclude $D_2 f$ is continuous in $\mathbb{R}^2$.
		\end{proof}
		
		\item
		\begin{proof}
			We know $D_{12} f(0,0) = 1$ and $D_{21} f(0,0) = -1$ from part(c).
			Calculate patiently, for $(x,y)\neq (0,0)$,
			$$
			D_{12} f(x,y) 
			= D_{21} f(x,y) 
			= \frac{x^6+9x^4y^2-9x^2y^4-y^6}{(x^2+y^2)^3}.
			$$
			Hence $D_{12} f$, $D_{21} f$ exist at every point of $\mathbb{R}^2$ and are continuous on $\mathbb{R}^2\setminus (0,0)$.
			
			In order to achieve they are not continuous at $(0,0$) indeed, taking the limit along $x=y$, we have
			$$
			\lim_{x\to 0}D_{12} f(x,x) 
			= \lim_{x\to 0} 0 
			= 0
			\neq D_{12} (0,0).
			$$
			A similar argument shows that $\lim_{x\to 0}D_{21} f(x,x) \neq D_{21}f(0,0)$.
			These mean $D_{12} f$, $D_{21} f$ are not continuous at $(0,0)$, as desired.
		\end{proof}
		
		\item
		\begin{proof}
			Calculate
			\begin{align*}
			D_{12} f(0,0)
			&= \lim_{h\to 0} \frac{D_2(h,0) - D_2(0,0)}{h}
			= \lim_{h\to 0} \frac{h}{h} 
			= 1; \\
			D_{21} f(0,0)
			&= \lim_{k\to 0} \frac{D_1(0,k) - D_1(0,0)}{k}
			= \lim_{k\to 0} \frac{-k}{k}
			= -1.
			\end{align*}
		\end{proof}
	\end{enumerate}
\end{Exercise}