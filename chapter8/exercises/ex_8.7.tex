% === Exercise 8.7 ===
\begin{Exercise}
	\begin{proof}
		Put $f(x) = \dfrac{2x}{\pi} - \sin x$.
		Since $f(0) = f(\dfrac{\pi}{2}) = 0$, by Rolle's Theorem, there exists $c\in\left(0, \dfrac{\pi}{2}\right)$ such that $f'(c) = 0$.
		Consider $f''(x) = \sin x > 0$ on $\left(0,\dfrac{\pi}{2}\right)$.
		This means $f'$ is strictly increasing on $\left(0,\dfrac{\pi}{2}\right)$.
		So, such $c$ is the only point, then $f(x) < 0$ for $0 < x < \dfrac{\pi}{2}$.
		That is,
		\begin{equation*}
		\frac{2}{\pi} < \frac{\sin x}{x}.\qquad\mbox{ $(0 < x < \frac{\pi}{2}$)}
		\end{equation*}
		
		On the other hand, put $g(x) = \sin x - x$.
		Then $g'(x) = \cos x - 1 < 0$ on $\left( 0, \dfrac{\pi}{2} \right)$.
		Therefore $g$ is strictly decreasing on this interval.
		Since $g(0) = 0$, we know $g(x) < 0$ on this interval.
		
		Hence we conclude for $0 < x < \dfrac{\pi}{2}$,
		$$
		\frac{2}{\pi} < \frac{\sin x}{x} < 1.
		$$
	\end{proof}
\end{Exercise}